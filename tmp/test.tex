\documentclass[a4paper]{article}

\usepackage{relations}
\usepackage{debugplastex}

\newtheorem{theorem}{Theorem}[section]
\newtheorem{lem}[theorem]{Lemma}
\newtheorem{exo}[theorem]{Exercise}

\renewcommand{\thecnttheorem}{\thesection .\arabic{cnttheorem}}

\title{Document test with long title}

\begin{document}
\section{Foundations}
\subsection{Axiomes}

$a = b$.

The main result follows from \ref{lem3}.
\subsection{Postulats}
\begin{equation}
\label{eqn:test}
a+b = ξ
\end{equation}

See\footnote{This is important} Equation \ref{eqn:test}.

\begin{lem}
	\label{lem1}
	Test de lemme.
\end{lem}

\section{Corollaries}

%\settrace
\begin{lem}
	\label{lem2}
	\uses{lem3}
	$1 + 1 = 2$.
\end{lem}


\begin{theorem}[Formule d'Euler]
\label{thm1}
\uses{lem1, lem2}
This is correct:
\[
	e^{i\pi} = -1.
\]
\end{theorem}

\begin{lem}
	\label{lem3}
	Lemma after theorem.
\end{lem}
\section{Exercices}

\begin{exo}
  \covers{lem1, thm1}
  Prove that $2 + 2 = 4$.
\end{exo}
\end{document}
