\documentclass{manual}

\usepackage[T1]{fontenc}
\usepackage{graphicx}
\usepackage{makeidx}
\usepackage{hyperref}

\newcommand{\titleref}{\ref}

\newcommand{\attr}[1]{\texttt{#1}}
\newcommand{\namespace}[1]{\texttt{#1:}}
\newcommand{\optval}[1]{\textrm{\textit{#1}}}
\newcommand{\macro}[1]{\textbackslash\texttt{#1}}
\newcommand{\environment}[1]{\texttt{#1}}
\newcommand{\question}{\subsection}
\newcommand{\LaTeXtohtml}{{\LaTeX}2html}
\newcommand{\plasTeX}{plas\TeX}
\newcommand{\configkeys}[2]{\texttt{#1:#2}}
\newenvironment{configuration}[1]{%
    \newcommand{\default}[1]{\textbf{Default:} ##1\\}%
    \newcommand{\config}[2]{\textbf{Config File:} [ ##1 ] ##2\\}%
    \newcommand{\options}[1]{\textbf{Command-Line Options:} %
                                     \texttt{##1}\\}%
    \begin{description}
    \item[\textbf{#1}] \hfill\\
}{\end{description}}

\title{plasTeX --- A Python Framework for Processing LaTeX Documents}
\author{Kevin D. Smith}
\authoraddress{\strong{SAS}\\Email: \email{Kevin.Smith@sas.com}}
\date{7 February 2008}

\makeindex
\makemodindex

\begin{document}

\maketitle
\cleardoublepage
\tableofcontents

\input{intro}


\chapter{\protect\program{plastex} --- The Command-Line Interface}
\label{sec:command-line}

While \plasTeX\ makes it possible to parse \LaTeX\ directly from Python
code, most people will simply use the supplied command-line interface, 
\program{plastex}.  \program{plastex} will invoke the parsing processes
and apply a specified renderer.  By default, \program{plastex} will
convert to HTML, although this can be changed in the \program{plastex}
configuration.  

Invoking \program{plastex} is very simple.  To convert a \LaTeX\ document 
to HTML using all of the defaults, simply type the following at shell prompt.

\begin{verbatim}
plastex mylatex.tex
\end{verbatim}

where \file{mylatex.tex} is the name of your \LaTeX\ file.  The 
\LaTeX\ source will be parsed, all packages will be loaded and macros 
expanded, and converted to HTML.  Hopefully, at this point you will have
a lovely set of HTML files that accurately reflect the \LaTeX\ source 
document.  Unfortunately, converting \LaTeX\ to other formats can be 
tricky, and there are many pitfalls.  If you are getting warnings or
errors while converting your document, you may want to check the FAQ
in the appendix to see if your problem is addressed.

Running \program{plastex} with the default options may not give you output
exactly the way you had envisioned.  Luckily, there are many options
that allow you to change the rendering behavior.  These options are 
described in the following section.


\section{Command-Line and Configuration Options}

There are many options to \program{plastex} that allow you to control
things input and output file encodings, where files are generated and
what the filenames look like, rendering parameters, etc.  While 
\program{plastex} is the interface where the options are specified, for
the most part these options are simply passed to the parser and renderers
for their use.  It is even possible to create your own options for use
in your own Python-based macros and renderers.
The following options are currently available
on the \program{plastex} command.  They are categorized for convenience. 

\subsection{General Options}\label{sec:general-options}

\begin{configuration}{Configuration files}
\options{\longprogramopt{config=\optval{config-file}} or
         \programopt{-c \optval{config-file}}}
\config{general}{config}
specifies a configuration file to load.  This should be the first option
specified on the command-line.
\end{configuration}

\begin{configuration}{Kpsewhich}
\options{\longprogramopt{kpsewhich=\optval{program}}}
\config{general}{kpsewhich}
\default{kpsewhich}
specifies the \program{kpsewhich} program to use to locate \LaTeX\
files and packages.
\end{configuration}

\begin{configuration}{Renderer}
\options{\longprogramopt{renderer=\optval{renderer-name}}}
\config{general}{renderer}
\default{XHTML}
specifies which renderer to use.
\end{configuration}

\begin{configuration}{Themes}
\options{\longprogramopt{theme=\optval{theme-name}}}
\config{general}{theme}
\default{default}
specifies which theme to use.
\end{configuration}

\begin{configuration}{Extra theme files}
\options{\longprogramopt{copy-theme-extras} or 
         \longprogramopt{ignore-theme-extras}}
\config{general}{copy-theme-extras}
\default{yes}
indicates whether or not extra files that belong to a theme (if there are
any) should be copied to the output directory.
\end{configuration}


\subsection{Document Properties\label{sec:config-document}}

\begin{configuration}{Base URL}
\options{\longprogramopt{base-url=\optval{url}}}
\config{document}{base-url}
specifies a base URL to prepend to the path of all links.
\end{configuration}

\begin{configuration}{Number of Columns in the Index}
\options{\longprogramopt{index-columns=\optval{integer}}}
\config{document}{index-columns}
specifies the number of columns to group the index into.
\end{configuration}

\begin{configuration}{Language terms}
\options{\longprogramopt{lang-terms=\optval{string}}}
\config{document}{lang-terms}
Specifies a list of files that contain language terms, delimited by the
OS path separator (such as : for POSIX and ; for Windows).
\end{configuration}

\begin{configuration}{Section number depth}
\options{\longprogramopt{sec-num-depth=\optval{integer}}}
\config{document}{sec-num-depth}
\default{6}
specifies the section level depth that should appear in section numbers.
This value overrides the value of the secnumdepth counter in the document.
\end{configuration}

\begin{configuration}{Title for the document}
\options{\longprogramopt{title=\optval{string}}}
\config{document}{title}
specifies a title to use for the document instead of the title given
in the \LaTeX\ source document
\end{configuration}

\begin{configuration}{Table of contents depth}
\options{\longprogramopt{toc-depth=\optval{integer}}}
\config{document}{toc-depth}
specifies the number of levels to include in each table of contents.
\end{configuration}

\begin{configuration}{Display sections in the table of contents that do not create files}
\options{\longprogramopt{toc-non-files}}
\config{document}{toc-non-files}
specifies that sections that do not create files should still appear in the
table of contents.  By default, only sections that create files will show
up in the table of contents.
\end{configuration}


\subsection{Counters}

It is possible to set the initial value of a counter from the
command-line using the \longprogramopt{counter} option or the
``counters'' section in a configuration file.  The configuration
file format for setting counters is very simple.  The option name
in the configuration file corresponds to the counter name, and the
value is the value to set the counter to.  
\begin{verbatim}
[counters]
chapter=4
part=2
\end{verbatim}

The sample configuration above sets the chapter counter to 4, and the
part counter to 2.

The \longprogramopt{counter} can also set counters.  It accepts multiple
arguments which must be surrounded by square brackets ([~]).
Each counter set in the \longprogramopt{counter}
option requires two values: the name of the counter and the value to
set the counter to.  An example of \longprogramopt{counter} is shown below.
\begin{verbatim}
plastex --counter [ part 2 chapter 4 ] file.tex
\end{verbatim}

Just as in the configuration example, this command-line sets the
part counter to 2, and the chapter counter to 4.

\begin{configuration}{Set initial counter values}
\options{\longprogramopt{counter=\optval{[ counter-name initial-value ]}}}
specifies the initial counter values.
\end{configuration}


\subsection{Document Links\label{sec:config-links}}

The links section of the configuration is a little different than the
others.  The options in the links section are not preconfigured, they
are all user-specified.  The links section includes information 
to be included in the navigation object available on all sections in 
a document.  By default, the section's navigation object includes things
like the previous and next objects in the document, the child nodes, 
the sibling nodes, etc.  The table below lists all of the navigation
objects that are already defined.  The names for these items came from
the link types defined at \url{http://fantasai.tripod.com/qref/Appendix/LinkTypes/ltdef.html}.  Of course, it is up to the renderer to actually make use
of them.

\begin{tableii}{l|l}{var}{Name}{Description}
\lineii{home}{the first section in the document}
\lineii{start}{same as \var{home}}
\lineii{begin}{same as \var{home}}
\lineii{first}{same as \var{home}}
\lineii{end}{the last section in the document}
\lineii{last}{same as \var{end}}
\lineii{next}{the next section in the document}
\lineii{prev}{the previous section in the document}
\lineii{previous}{same as \var{prev}}
\lineii{up}{the parent section}
\lineii{top}{the top section in the document}
\lineii{origin}{same as \var{top}}
\lineii{parent}{the parent section}
\lineii{child}{a list of the subsections}
\lineii{siblings}{a list of the sibling sections}
\lineii{document}{the document object}
\lineii{part}{the current part object}
\lineii{chapter}{the current chapter object}
\lineii{section}{the current section object}
\lineii{subsection}{the current subsection object}
\lineii{navigator}{the top node in the document object}
\lineii{toc}{the node containing the table of contents}
\lineii{contents}{same as \var{toc}}
\lineii{breadcrumbs}{a list of the parent objects of the current node}
\end{tableii}

Since each of these items references an object that is expected to have
a URL and a title, any user-defined fields should contain these as well
(although the URL is optional in some items).  To create a user-defined
field in this object, you need to use two options: one for the title
and one for the URL, if one exists.  They are specified in the config
file as follows:
\begin{verbatim}
[links]
next-url=http://myhost.com/glossary
next-title=The Next Document
mylink-title=Another Title
\end{verbatim}

These option names are split on the dash (-) to create a key, before the dash,
and a member, after the dash.  A dictionary is inserted into the navigation
object with the name of the key, and the members are added to that dictionary.
The configuration above would create the following Python dictionary.
\begin{verbatim}
{
    'next':
        {
            'url':'http://myhost.com/glossary', 
            'title':'The Next Document'
        },
    'mylink':
        {
            'title':'Another Title'
        }
}
\end{verbatim}

While you can not override a field that is populated by the document, 
there are times when a field isn't populated.  This occurs, for example,
in the \var{prev} field at the beginning of the document, or the 
\var{next} field at the end of the document.  If you specify a \var{prev}
or \var{next} field in your configuration, those fields will be used
when no \var{prev} or \var{next} is available.  This allows you to link
to external documents at those points.


\begin{configuration}{Set document links}
\options{\longprogramopt{links=\optval{[ key optional-url title ]}}}
specifies links to be included in the navigation object.  Since at
least two values are needed in the links (key and title, with an optional
URL), the values are grouped in square brackets on the command-line ([~]).
\end{configuration}


\subsection{Input and Output Files\label{sec:config-files}}

If you have a renderer that only generates one file, specifying the output
filename is simple: use the \longprogramopt{filename} option to specify
the name.  However, if the renderer you are using generates multiple
files, things get more complicated.  The \longprogramopt{filename} option
is also capable of handling multiple names, as well as giving you a
templating way to build filenames.

Below is a list of all of the options that affect filename generation.

\begin{configuration}{Characters that shouldn't be used in a filename}
\options{\longprogramopt{bad-filename-chars=\optval{string}}}
\config{files}{bad-chars}
\default{:~\#\$\%\textasciicircum\&*!\textasciitilde`"'=?/{}[]()|<>;\textbackslash,.}
specifies all characters that should not be allowed in a filename.
These characters will be replaced by the value in 
\longprogramopt{bad-filename-chars-sub}.
\end{configuration}

\begin{configuration}{String to use in place of invalid characters}
\options{\longprogramopt{bad-filename-chars-sub}=\optval{string}}
\config{files}{bad-chars-sub}
\default{-}
specifies a string to use in place of invalid filename characters (
specified by the \longprogramopt{bad-chars-sub} option)
\end{configuration}

\begin{configuration}{Output Directory}
\options{\longprogramopt{dir=\optval{directory}}  or \programopt{-d \optval{directory}}}
\config{files}{directory}
\default{\$jobname}
specifies a directory name to use as the output directory.
\end{configuration}

\begin{configuration}{Escaping characters higher than 7-bit}
\options{\longprogramopt{escape-high-chars}}
\config{files}{escape-high-chars}
\default{False}
some output types allow you to represent characters that are greater than
7-bits with an alternate representation to alleviate the issue of 
file encoding.  This option indicates that these alternate representations
should be used.

\note{The renderer is responsible for doing the translation into the 
alternate format.  This might not be supported by all output types.}
\end{configuration}

\begin{configuration}{Template to use for output filenames}
\options{\longprogramopt{filename=\optval{string}}}
\config{files}{filename}
specifies the templates to use for generating filenames.  
The filename template is a list of space separated names.  Each name 
in the list is returned once.  An example is shown below.

\begin{verbatim}
index.html toc.html file1.html file2.html
\end{verbatim}

If you don't know how many files you are going to be reproducing,
using static filenames like in the example above is not practical.
For this reason, these filenames can also contain variables as described in 
Python's string Templates (e.g. \var{\$title}, \var{\${id}}).  These variables
come from the namespace created in the renderer and include:
\var{\$id}, the ID (i.e. label) of the item, \var{\$title}, the title of the
item, and \var{\$jobname}, the basename of the \LaTeX\ file being processed.  
One special variable is \var{\$num}.  This value in generated dynamically 
whenever a filename with \var{\$num} is requested.  Each time a filename 
with \var{\$num} is successfully generated, the value of \var{\$num}
is incremented.

The values of variables can also be modified by a format specified
in parentheses after the variable.  The format is simply an integer
that specifies how wide of a field to create for integers 
(zero-padded), or, for strings, how many space separated words
to limit the name to.  The example below shows \var{\$num} being padded
to four places and \var{\$title} being limited to five words.

\begin{verbatim}
sect$num(4).html $title(5).html
\end{verbatim}

The list can also contain a wildcard filename (which should be 
specified last).  Once a wildcard name is reached, it is 
used from that point on to generate the remaining filenames.  
The wildcard filename contains a list of alternatives to use as
part of the filename indicated by a comma separated list of 
alternatives surrounded by a set of square brackets ([ ]).
Each of the alternatives specified is tried until a filename is
successfully created (i.e. all variables resolve).  For example,
the specification below creates three alternatives.

\begin{verbatim}
$jobname_[$id, $title, sect$num(4)].html
\end{verbatim}

The code above is expanded to the following possibilities.

\begin{verbatim}
$jobname_$id.html
$jobname_$title.html
$jobname_sect$num(4).html
\end{verbatim}

Each of the alternatives is attempted until one of them succeeds.
In order for an alternative to succeed, all of the variables referenced
in the template must be populated.  For example, the \var{\$id} variable
will not be populated unless the node had a \macro{\$label} macro
pointing to it.  The \var{title} variable would not be populated unless
the node had a title associated with it (e.g. such as section, subsection, etc.).
Generally, the last one should contain no variables except for
\var{\$num} as a fail-safe alternative.
\end{configuration}

\begin{configuration}{Input Encoding}
\options{\longprogramopt{input-encoding=\optval{string}}}
\config{files}{input-encoding}
\default{utf-8}
specifies which encoding the \LaTeX\ source file is in
\end{configuration}

\begin{configuration}{Output Encoding}
\options{\longprogramopt{output-encoding=\optval{string}}}
\config{files}{output-encoding}
\default{utf-8}
specifies which encoding the output files should use.  
\note{This depends on the output format as well.  While HTML and XML use 
encodings, a binary format like MS Word, would not.}
\end{configuration}

\begin{configuration}{Splitting document into multiple files}
\options{\longprogramopt{split-level=\optval{integer}}}
\config{files}{split-level}
\default{2}
specifies the highest section level that generates a new file.  Each section
in a \LaTeX\ document has a number associated with its hierarchical level.
These levels are -2 for the document, -1 for parts, 0 for chapters,
1 for sections, 2 for subsections, 3 for subsubsections, 4 for paragraphs,
and 5 for subparagraphs.  A new file will be generated for every section
in the hierarchy with a value less than or equal to the value of this 
option.  This means that for the value of 2, files will be generated for
the document, parts, chapters, sections, and subsections.
\end{configuration}


\subsection{Image Options\label{sec:config-images}}

Images are created by renderers when the output type in incapable of 
rendering the content in any other way.  This method is commonly used
to display equations in HTML output.  The following options control
how images are generated.

\begin{configuration}{Base URL}
\options{\longprogramopt{image-base-url=\optval{url}}}
\config{images}{base-url}
specifies a base URL to prepend to the path of all images.
\end{configuration}

\begin{configuration}{\LaTeX\ program to use to compile image document}
\options{\longprogramopt{image-compiler=\optval{program}}}
\config{images}{compiler}
\default{latex}
specifies which program to use to compile the images \LaTeX\ document.
\end{configuration}

\begin{configuration}{Enable or disable image generation}
\options{\longprogramopt{enable-images} or
         \longprogramopt{disable-images}}
\config{images}{enabled}
\default{yes}
indicates whether or not images should be generated.
\end{configuration}

\begin{configuration}{Enable or disable the image cache}
\options{\longprogramopt{enable-image-cache} or
         \longprogramopt{disable-image-cache}}
\config{images}{cache}
\default{yes}
indicates whether or not images should use a cache between runs.
\end{configuration}

\begin{configuration}{Convert \LaTeX\ output to images}
\options{\longprogramopt{imager=\optval{program}}}
\config{images}{imager}
\default{dvipng dvi2bitmap gsdvipng gspdfpng OSXCoreGraphics}
specifies which converter will be used to take the output from the
\LaTeX\ compiler and convert it to images.  You can specify a space 
delimited list of names as well.  If a list of names is specified, 
each one is verified in order to see if it works on the current machine.
The first one that succeeds is used.

You can use the value of ``none'' to turn the imager off.
\end{configuration}

\begin{configuration}{Image filenames}
\options{\longprogramopt{image-filenames=\optval{filename-template}}}
\config{images}{filenames}
\default{images/img-\$num(4).png}
specifies the image naming template to use to generate filenames.  This
template is the same as the templates used by the \longprogramopt{filename}
option.
\end{configuration}

\begin{configuration}{Convert \LaTeX\ output to vector images}
\options{\longprogramopt{vector-imager=\optval{program}}}
\config{images}{vector-imager}
\default{dvisvgm}
specifies which converter will be used to take the output from the
\LaTeX\ compiler and convert it to vector images.  You can specify a space 
delimited list of names as well.  If a list of names is specified, 
each one is verified in order to see if it works on the current machine.
The first one that succeeds is used.

You can use the value of ``none'' to turn the vector imager off.

\note{When using the vector imager, a bitmap image is also created
using the regular imager.  This bitmap is used to determine the 
depth information about the vector image and can also be used as
a backup if the vector image is not supported by the viewer.}
\end{configuration}


\input{document}

\input{macros}


\chapter{Renderers}

Renderers allow you to convert a \plasTeX\ document object into viewable
output such as HTML, RTF, or PDF, or simply a data structure format such
as DocBook or tBook.  Since the \plasTeX\ document object gives you 
everything that you could possibly want to know about the \LaTeX\ document,
it should, in theory, be possible to generate any type of output from
the \plasTeX\ document object while preserving as much information as the
output format is capable of.  In addition, since the document object is
not affected by the rendering process, you can apply multiple renderers
in sequence so that the \LaTeX\ document only needs to be parsed one time
for all output types.

While it is possible to write a completely custom renderer, one possible
rendeerer implementation is included with the \plasTeX\ framework.
While the rendering process in this implementation is fairly simple,
it is also very powerful.  Some of the main features are listed below.
\begin{itemize}
\item ability to generate multiple output files
\item automatic splitting of files is configurable by section level,
    or can be invoked using ad-hoc methods in the 
    \member{filenameoverride} property
\item powerful output filename generation utility
\item image generation for portions of the document that cannot be 
    easily rendered in a particular output formate (e.g. equations in HTML)
\item themeing support
\item hooks for post-processing of output files
\item configurable output encodings
\end{itemize}

The API of the renderer itself is very small.  In fact, there are only
a couple of methods that are of real interest to an end user: \method{render}
and \method{cleanup}.  The \method{render} method is the method that starts
the rendering process.  It's only argument is a \plasTeX\ document object.
The \method{cleanup} method is called at the end of the rendering process.
It is passed the document object and a list of all of the files that were
generated.  This method allows you to do post-processing on the output files.
In general, this method will probably only be of interest to someone 
writing a subclass of the \class{Renderer} class, so most users of 
\plasTeX\ will only use the \method{render} method.  The real work of
the rendering process is handled in the \class{Renderable} class which
is discussed later in this chapter.

The \class{Renderer} class is a subclass of the Python dictionary.  
Each key in the renderer corresponds to the name of a node in the 
document object.  The value stored under each key is a function.
As each node in the document object is traversed, the renderer is queried 
to see if there is a key that matches the name of the node.  If a 
key is found, the value at that key (which must be a function) is 
called with the node as its only argument.  The return value from this
call must be a unicode object that contains the rendered output.
Based on the configuration, the renderer will handle all of the file 
generation and encoding issues.

If a node is traversed that doesn't correspond to a key in the 
renderer dictionary, the default rendering method is called.  The 
default rendering method is stored in the \member{default} attribute.
One exception to this rule is for text nodes.  The default 
rendering method for text nodes is actually stored in \member{textDefault}.
Again, these attributes simply need to reference any Python function
that returns a unicode object of the rendered output.  The 
default method in both of these attributes is the \function{unicode}
built-in function.

As mention previously, most of the work of the renderer is actually 
done by the \class{Renderable} class.  This is a mixin class\footnote{
A mixin class is simply a class that is merely a collection of methods
that are intended to be included in the namespace of another class.} that is
mixed into the \class{Node} class in the \method{render} method.
It is unmixed at the end of the \method{render} method.  The details
of the \class{Renderable} class are discussed in section \ref{sec:renderable}.

\section{Simple Renderer Example}\label{sec:simple-renderer-ex}

It is possible to write a renderer with just a couple of methods:
\method{default} and \method{textDefault}.
The code below demonstrates how one might create a generic XML
renderer that simply uses the node names as XML tag names.  
The text node renderer escapes the <, >, and \& characters.
\begin{verbatim}
import string
from plasTeX.Renderers import Renderer

class Renderer(Renderer):
    
    def default(self, node):
        """ Rendering method for all non-text nodes """
        s = []

        # Handle characters like \&, \$, \%, etc.
        if len(node.nodeName) == 1 and node.nodeName not in string.letters:
            return self.textDefault(node.nodeName)

        # Start tag
        s.append('<%s>' % node.nodeName)

        # See if we have any attributes to render
        if node.hasAttributes():
            s.append('<attributes>')
            for key, value in node.attributes.items():
                # If the key is 'self', don't render it
                # these nodes are the same as the child nodes
                if key == 'self':
                    continue
                s.append('<%s>%s</%s>' % (key, unicode(value), key))
            s.append('</attributes>')

        # Invoke rendering on child nodes
        s.append(unicode(node))

        # End tag
        s.append('</%s>' % node.nodeName)

        return u'\n'.join(s)

    def textDefault(self, node):
        """ Rendering method for all text nodes """
        return node.replace('&','&amp;').replace('<','&lt;').replace('>','&gt;')
\end{verbatim}

To use the renderer, simply parse a \LaTeX\ document and apply the renderer
using the \method{render} method.
\begin{verbatim}
# Import renderer from previous code sample
from MyRenderer import Renderer

from plasTeX.TeX import TeX

# Instantiate a TeX processor and parse the input text
tex = TeX()
tex.ownerDocument.config['files']['split-level'] = -100
tex.ownerDocument.config['files']['filename'] = 'test.xml'
tex.input(r'''
\documentclass{book}
\begin{document}

Previous paragraph.

\section{My Section}

\begin{center}
Centered text with <, >, and \& charaters.
\end{center}

Next paragraph.

\end{document}
''')
document = tex.parse()

# Render the document
renderer = Renderer()
renderer.render(document)
\end{verbatim}

The output from the renderer, located in \file{test.xml}, looks like the
following.
\begin{verbatim}
<document>
<par>
Previous paragraph.
</par><section>
    <attributes>
        <toc>None</toc>
        <*modifier*>None</*modifier*>
        <title>My Section</title>
    </attributes>
<par>
<center>
 Centered text with &lt;, &gt;, and &amp; charaters.
</center>
</par><par>
Next paragraph.
</par>
</section>
</document>
\end{verbatim}


\subsection{Extending the Simple Renderer}

Now that we have a simple renderer working, it is very simple to extend
it to do more specific operations.  Let's say that the default renderer
is fine for most nodes, but for the \macro{section} node we want to do
something special.  For the section node, we want the title argument
to correspond to the title attribute in the output XML\footnote{This
will only work properly in XML if the content of the title is plain text
since other nodes will generate markup.}.  To do this we need a 
method like the following.
\begin{verbatim}
def handle_section(node):
    return u'\n\n<%s title="%s">\n%s\n</%s>\n' % \
            (node.nodeName, unicode(node.attributes['title']), 
             unicode(node), node.nodeName)
\end{verbatim}

Now we simply insert the rendering method into the renderer under the
appropriate key.  Remember that the key in the renderer should match
the name of the node you want to render.  Since the above rendering 
method will work for all section types, we'll insert it into the 
renderer for each \LaTeX\ sectioning command.
\begin{verbatim}
renderer = Renderer()
renderer['section'] = handle_section
renderer['subsection'] = handle_section
renderer['subsubsection'] = handle_section
renderer['paragraph'] = handle_section
renderer['subparagraph'] = handle_section
renderer.render(document)
\end{verbatim}

Running the same \LaTeX\ document as in the previous example, we now get
this output.
\begin{verbatim}
<document>
<par>
Previous paragraph.
</par>

<section title="My Section">
<par>
<center>
 Centered text with &lt;, &gt;, and &amp; charaters.
</center>
</par><par>
Next paragraph.
</par>
</section>

</document>
\end{verbatim}

Of course, you aren't limited to using just Python methods.  Any function
that accepts a node as an argument can be used.  The 
Zope Page Template (ZPT) renderer included with \plasTeX\ is an example 
of how to write a renderer that uses a templating language to render
the nodes (see section \ref{sec:zpt}).

\subsection{Using a Renderer from the plastex Script}

In the preceding sections, the simple renderer example was called from
a custom python script. In order to use it through the
main plastex script (described in Chapter~\ref{sec:command-line}), it
needs to be located in some directory
\verb+plasTeX/Renderers/SimpleRenderer+, where \verb+plasTeX+ is the
directory containing the plastex script. This directory must contain a
\verb+__init__.py+ file defining the \var{Renderer} class (with this
name). This directory can also contain a \verb+Themes+ directory in
order to use the theme option described in
Section~\ref{sec:general-options}. Each subdirectory in the
\verb+Themes+ directory is considered as a theme.

Each renderer can define its own configuration options which are loaded
by the plastex script. This is done in a file named \verb+Config.py+
in the renderer directory. This file must define a variable named
\var{config} which is a ConfigManager instance, as described in
Section~\ref{sec:configuration-api}. Inspiration can be drawn from the
file defining the global configuration which is
\verb+plasTeX/Config.py+.

For instance, one could add a file
\verb+plasTeX/Renderers/SimpleRenderer/Config.py+ containing:

\begin{verbatim}
from plasTeX.ConfigManager import *

config = ConfigManager()

section = config.add_section('simplerenderer')

config.add_category('simplerenderer', 'Simple Renderer Options')

section['my-option'] = StringOption(
    """ My option """,
    options='--my-option',
    category='simplerenderer',
    default='',
)
\end{verbatim}

Options values are attached to the document currently rendered. For
instance, in the \var{default} method implemented in
Section~\ref{sec:simple-renderer-ex}, which takes a node argument, one
could access the value of the option defined above as
\verb+node.ownerDocument.config['simplerenderer']['my-option']+.


\section{Renderable Objects\label{sec:renderable}}

The \class{Renderable} class is the real workhorse of the rendering process.
It traverses the document object, looks up the appropriate rendering
methods in the renderer, and generates the output files.  It also 
invokes the image generating process when needed for parts of a document
that cannot be rendered in the given output format.

Most of the work of the \class{Renderable} class is done in the 
\method{__unicode__} method.  This is rather convenient since each of
the rendering methods in the renderer are required to return a unicode
object.  When the \function{unicode} function is called with a renderable
object as its argument, the document traversal begins for that node.
This traversal includes iterating through each of the node's child nodes, and
looking up and calling the appropriate rendering method in the renderer.
If the child node is configured to generate a new output file, the 
file is created and the rendered output is written to it; otherwise,
the rendered output is appended to the rendered output of previous nodes.
Once all of the child nodes have been rendered, the unicode object containing
that output is returned.  This recursive process continues until the 
entire document has been rendered.

There are a few useful things to know about renderable objects such as
how they determine which rendering method to use, when to generate new
files, what the filenames will be, and how to generate images.  These
things are discussed below.


\subsection{Determining the Correct Rendering Method}

Looking up the correct rendering method is quite straight-forward.  
If the node is a text node, the \member{textDefault} attribute on 
the renderer is used.  If it is not a text node, then the node's name
determines the key name in the renderer.  In most cases, the node's
name is the same name as the \LaTeX\ macro that created it.  If the
macro used some type of modifier argument (i.e. *, +, -), a name 
with that modifier applied to it is also searched for first.  For example,
if you used the \environment{tablular*} environment in your \LaTeX\
document, the renderer will look for ``tabular*'' first, then ``tabular''.
This allows you to use different rendering methods for modified and
unmodified macros.  If no rendering method is found, the method
in the renderer's \member{default} attribute is used.


\subsection{Generating Files}

Any node in a document has the ability to generate a new file. 
During document traversal, each node is queried for a filename.  If
a non-\var{None} is returned, a new file is created for the content
of that node using the given filename.  The querying for the filename
is simply done by accessing the \member{filename} property of the
node.  This property is added to the node's namespace during the
mixin process.  The default behavior for this property is to only
return filenames for sections with a level less than the split-level
given in the configuration (see section \ref{sec:config-files}).
The filenames generated by this routine are very flexible.  They can 
be statically given names, or names based on the ID and/or title, 
or simply generically numbered.  For more information on configuring
filenames see section \ref{sec:config-files}.

While the filenaming mechanism is very powerful, you may want to give
your files names based on some other information.  This is possible through
the \member{filenameoverride} attribute.  If the \member{filenameoverride}
is set, the name returned by that attribute is used as the filename.
The string in \member{filenameoverride} is still processed in the same
way as the filename specifier in the configuration so that you can 
use things like the ID or title of the section in the overridden filename.

The string used to specify filenames can also contain directory paths.
This is not terribly useful at the moment since there is no way to 
get the relative URLs between two nodes for linking purposes.

If you want to use a filename override, but want to do it conditionally
you can use a Python property to do this.  Just calculate the filename
however you wish, if you decide that you don't want to use that filename
then raise an \exception{AttributeError} exception.  An example of this
is shown below.
\begin{verbatim}
class mymacro{Command):
    args = '[ filename:str ] self'
    @property
    def filenameoverride(self):
        # See if the attributes dictionary has a filename
        if self.attributes['filename'] is not None:
            return self.attributes['filename']
        raise AttributeError, 'filenameoverride'
\end{verbatim}
\note{The filename in the \member{filenameoverride} attribute must 
   contain any directory paths as well as a file extension.}


\subsection{Generating Images}

Not all output types that you might render are going to support everything
that \LaTeX\ is capable of.  For example, HTML has no way of representing
equations, and most output types won't be capable of rendering 
\LaTeX's \environment{picture} environment.  In cases like these, you
can let \plasTeX\ generate images of the document node.  Generating
images is done with a subclass of \class{plasTeX.Imagers.Imager}.
The imager is responsible for creating a \LaTeX\ document from the
requested document fragments, compiling the document and converting
each page of the output document into individual images.  Currently,
there are two \class{Imager} subclasses included with \plasTeX.
Each of them use the standard \LaTeX\ compiler to generate a DVI file.
The DVI file is then converted into images using one of the available
imagers (see section \ref{sec:config-images} on how to select different imagers).

To generate an image of a document node, simply access the \member{image}
property during the rendering process.  This property will return
an \class{plasTeX.Imagers.Image} instance.  In most cases, the image
file will not be available until the rendering process is finished
since most renderers will need the generated \LaTeX\ document to be
complete before compiling it and generating the final images.

The example below demonstrates how to generate an image for the 
\environment{equation} environment.
\begin{verbatim}
# Import renderer from first renderer example
from MyRenderer import Renderer

from plasTeX.TeX import TeX

def handle_equation(node):
    return u'<div><img src="%s"/></div>' % node.image.url

# Instantiate a TeX processor and parse the input text
tex = TeX()
tex.input(r'''
\documentclass{book}
\begin{document}

Previous paragraph.

\begin{equation}
\Sigma_{x=0}^{x+n} = \beta^2
\end{equation}

Next paragraph.

\end{document}
''')
document = tex.parse()

# Instantiate the renderer
renderer = Renderer()

# Insert the rendering method into all of the environments that might need it
renderer['equation'] = handle_equation
renderer['displaymath'] = handle_equation
renderer['eqnarray'] = handle_equation

# Render the document
renderer.render(document)
\end{verbatim}

The rendered output looks like the following, and the image is generated
is located in \file{images/img-0001.png}.
\begin{verbatim}
<document>
<par>
Previous paragraph.
</par><par>
<div><img src="images/img-0001.png"/></div>
</par><par>
Next paragraph.
</par>
</document>
\end{verbatim}

The names of the image files are determined by the document's configuration.
The filename generator is very powerful, and is in fact, the same filename
generator used to create the other output filenames.  For more information
on customizing the image filenames see section \ref{sec:config-images}.

In addition, the image types are customizable as well.  \plasTeX\ uses
the Python Imaging Library (PIL) to do the final cropping and saving of the
image files, so any image format that PIL supports can be used.  The
format that PIL saves the images in is determined by the file extension
in the generated filenames, so you must use a file extension that 
PIL recognizes.

It is possible to write your own \class{Imager} subclass if necessary.
See the \class{Imager} API documentation for more information (see
\ref{sec:imager-api}).


\subsection{Generating Vector Images}

If you have a vector imager configured (such as dvisvg or dvisvgm), you
can generate a vector version of the requested image as well as a 
bitmap.  The nice thing about vector versions of images is that they
can scale infinitely and not loose resolution.  The bad thing about them
is that they are not as well supported in the real world as bitmaps.

Generating a vector image is just as easy as generating a bitmap image,
you simply access the \member{vectorImage} property of the node that
you want an image of.  This will return an \class{plasTeX.Imagers.Image} 
instance that corresponds to the vector image.  A bitmap version of 
the same image can be accessed through the \member{image} property of the
document node or the \member{bitmap} variable of the vector image object.

Everything that was described about generating images in the previous 
section is also true of vector images with the exception of cropping.
\plasTeX\ does not attempt to crop vector images.  The program that
converts the \LaTeX\ output to a vector image is expected to crop the 
image down to the image content.  \plasTeX\ uses the information from
the bitmap version of the image to determine the proper depth of the 
vector image.


\subsection{Static Images}

There are some images in a document that don't need to be generated, they
simply need to be copied to the output directory and possibly converted
to an appropriate formate.  This is accomplished with the 
\member{imageoverride} attribute.  When the \member{image} property
is accessed, the \member{imageoverride} attribute is checked to see if
an image is already available for that node.  If there is, the image
is copied to the image output directory using a name generated 
using the same method as described in the previous section.  The image
is copied to that new filename and converted to the appropriate 
image format if needed.  While it would be possible to simply copy the
image over using the same filename, this may cause filename collisions
depending on the directory structure that the original images were
store in.

Below is an example of using \member{imageoverride} for copying 
stock icons that are used throughout the document.
\begin{verbatim}
from plasTeX import Command

class dangericon(Command):
    imageoverride = 'danger.gif'

class warningicon(Command):
    imageoverride = 'warning.gif'
\end{verbatim}

It is also possible to make \member{imageoverride} a property
so that the image override can done conditionally.  In the case
where no override is desired in a property implementation, simply
raise an \exception{AttributeError} exception.

\section{Page Template Renderer\label{sec:zpt}}

The Page Template (PT) renderer is a renderer for \plasTeX\ document
objects that supports various page template engines such as 
\href{http://www.zope.org/Documentation/Books/ZopeBook/2_6Edition/ZPT.stx}{Zope
Page Templates (ZPT)}, \href{http://jinja.pocoo.org/}{Jinja2 templates},
\href{http://www.cheetahtemplate.org/}{Cheetah templates}, 
\href{http://kid-templating.org/}{Kid templates}, 
\href{http://genshi.edgewall.org/}{Genshi templates}, 
\href{http://docs.python.org/lib/node40.html}{Python string templates}, 
as well as plain old \href{http://docs.python.org/lib/typesseq-strings.html}{Python string formatting}.  It is also possible to add support for other 
template engines.  Note that all template engines except ZPT, Python formats, 
and Python string templates must be installed in your Python installation. 
They are not included.

ZPT is the most supported page template language at the moment.  This is the
template engine that is used for all of the \plasTeX\ delivered templates
in the XHTML renderer; however, the other templates work in a very similar way.
The actual ZPT implementation used is SimpleTAL 
(\url{http://www.owlfish.com/software/simpleTAL/}).  This implementation
implements almost all of the ZPT API and is very stable.  However, some
changes were made to this package to make it more convenient to use
within \plasTeX.  These changes are discussed in detail in the 
ZPT Tutorial (see section \ref{sec:zpttutorial}).

Since the above template engines can be used to generate any form of 
XML or HTML, the PT
renderer is a general solution for rendering XML or HTML from a 
\plasTeX\ document object.  When switching from one DTD to another, 
you simply need to use a different set of templates.

As in all \class{Renderer}-based renderers, each key in the PT renderer
returns a function.  These functions are actually generated when the 
template files are parsed by the PT renderer.
As is the case with all rendering methods, the only argument is the node to be
rendered, and the output is a unicode object containing the rendered 
output. In addition to the rendering methods, the \method{textDefault} method
escapes all characters that are special in XML and HTML (i.e. <, >, and \&).

The following sections describe how templates are loaded into the 
renderer, how to extend the set of templates with your own, as well
as a theming mechanism that allows you to apply different looks to 
output types that are visual (e.g. HTML).

\subsection{Defining and Using Templates}

\note{If you are not familiar with the ZPT language, you should read the 
tutorial in section \ref{sec:zpttutorial} before continuing in this 
section.  See the links in the previous section for documentation on
the other template engines.}

By default, templates are loaded from the directory where the 
renderer module was imported from.  In addition, the templates from
each of the parent renderer class modules are also loaded.  This makes
it very easy to extend a renderer and add just a few new templates
to support the additions that were made.

The template files in the module directories can have three different forms.
The first is HTML.  HTML templates must have an extension of \file{.htm} or
 \file{.html}.  These templates are compiled using 
SimpleTAL's HTML compiler.  XML templates, the second form of template, 
uses SimpleTAL's XML compiler, so they must be well-formed XML 
fragments.  XML templates must have the file extension \file{.xml}, 
\file{.xhtml}, or \file{.xhtm}.  In any case, the basename of the template
file is used as the key to store the template in the renderer.  Keep in
mind that the names of the keys in the renderer correspond to the node
names in the document object.  

The extensions used for all templating engines are shown in the table below.
\begin{tableiii}{l|l|l}{}{Engine}{Extension}{Output Type}
\lineiii{ZPT}{.html, .htm, .zpt}{HTML}
\lineiii{}{.xhtml, .xhtm, .xml}{XML/XHTML}
\lineiii{Jinja2}{.jinja2}{Any}
\lineiii{Python string formatting}{.pyt}{Any}
\lineiii{Python string templates}{.st}{Any}
\lineiii{Kid}{.kid}{XML/XHTML}
\lineiii{Cheetah}{.che}{XML/XHTML}
\lineiii{Genshi}{.gen}{HTML}
\end{tableiii} 

The file listing below is an example of a directory of template files.
In this case the templates correspond to nodes in the document created
by the \environment{description} environment, the \environment{tabular}
environment, \macro{textbf}, and \macro{textit}.
\begin{verbatim}
description.xml
tabular.xml
textbf.html
textit.html
\end{verbatim}

Since there are a lot of templates that are merely one line, it would be
inconvenient to have to create a new file for each template.  In cases
like this, you can use the \file{.zpts} extension for collections of
ZPT templates, or the \file{.jinja2s} extension for collections of
Jinja2 templates, or more generally \file{.pts} for collections of various 
template types.  Files with this
extension have multiple templates in them.  Each template is separated
from the next by the template metadata which includes things like the 
name of the template, the type (xml, html, or text), and can also alias 
template names to another template in the renderer.  The following 
metadata names are currently supported.
\begin{tableii}{l|p{4in}}{}{Name}{Purpose}
\lineii{engine}{the name of the templating engine to use.  At the time of
    this writing, the value could be zpt, tal (same as zpt), 
    html (ZPT HTML template), xml (ZPT XML template), jinja2,
		python (Python formatted string), string (Python string template),
    kid, cheetah, or genshi.}
\lineii{name}{the name or names of the template that is to follow. 
    This name is used as the key in the renderer, and also 
    corresponds to the node name that will be rendered by the template.
    If more than one name is desired, they are simply separated by
    spaces.}
\lineii{type}{the type of the template: xml, html, or text.  XML templates
    must contain a well-formed XML fragment.  HTML templates are more 
    forgiving, but do not support all features of ZPT (see the SimpleTAL
    documentation).}
\lineii{alias}{specifies the name of another template that the given
    names should be aliased to.  This allows you to simply reference
    another template to use rather than redefining one.  For example,
    you might create a new section heading called \macro{introduction}
    that should render the same way as \macro{section}.  In this case,
    you would set the name to ``introduction'' and the alias to
    ``section''.}
\end{tableii}

There are also some defaults that you can set at the top of the file that
get applied to the entire file unles overridden by the meta-data on a 
particular template.
\begin{tableii}{l|p{4in}}{}{Name}{Purpose}
\lineii{default-engine}{the name of the engine to use for all templates in
    the file.}
\lineii{default-type}{the default template type for all templates in the file.}
\end{tableii}

The code sample below shows the basic format of a zpts file.
\begin{verbatim}
name: textbf bfseries
<b tal:content="self">bold content</b>

name: textit
<i tal:content="self">italic content</i>

name: introduction introduction*
alias: section

name: description
type: xml
<dl>
<metal:block tal:repeat="item self">
    <dt tal:content="item/attributes/term">definition term</dt>
    <dd tal:content="item">definition content</dd>
</metal:block>
</dl>
\end{verbatim}

The code above is a zpts file that contains four templates.  Each template
begins when a line starts with ``name:''.  Other directives have the same
format (i.e. the name of the directive followed by a colon) and must 
immediately follow the name directive.  The first template definition 
actually applies to two types of nodes \var{textbf} and \var{bfseries}.
You can specify ony number of names on the name line.  The third template
isn't a template at all; it is an alias.  When an alias is specified,
the name (or names) given use the same template as the one specified 
in the alias directive.  Notice also that starred versions of a macro
can be specified separately.  This means that they can use a different
template than the un-starred versions of the command.
The last template is just a simple XML formatted
template.  By default, templates in a zpts file use the HTML compiler
in SimpleTAL.  You can specify that a template is an XML template by using
the type directive.

Here is an example of using various templates engines in a single file.
\begin{verbatim}
name: equation
engine: jinja2
<div class="equation" id="{{ obj.id }}">
  <span class="equation_label">{{ obj.ref }}</span>
  {{ obj }}
</div>

name: textbf
engine: python
<b>%(self)s</b>

name: textit
engine: string
<i>${self}</i>

name: textsc
engine: cheetah
<span class="textsc">${here}</span>

name: textrm
engine: kid
<span class="textrm" py:content="XML(unicode(here))">normal text</span>

name: textup
engine: genshi
<span class="textup" py:content="markup(here)">upcase text</span>
\end{verbatim}

There are several variables inserted into the template namespace.  Here is
a list of the variables and the templates that support them.

\begin{center}
\begin{tabular}{|l|l|l|l|l|}\hline
\textbf{Object} & \textbf{ZPT/Python Formats/String Template} &
\textbf{Jinja2} &
    \textbf{Cheetah} & \textbf{Kid/Genshi}\\\hline
document node & \var{self} or \var{here} & \var{obj} or \var{here} &  \var{here} & \var{here} \\
parent node & \var{container} & \var{container} & \var{container} & \var{container} \\
document config & \var{config} & \var{config} & \var{config} & \var{config} \\
template instance & \var{template} &  & & \\
renderer instance & \var{templates} & \var{templates} & \var{templates} & \var{templates} \\\hline
\end{tabular}
\end{center}

You'll notice that Kid and Genshi templates require some extra processing
of the variables in order to get the proper markup.  By default, these templates
escape characters like <, >, and \&.  In order to get HTML/XML markup from
the variables you must wrap them in the code shown in the example above.
Hopefully, this limitation will be removed in the future.


\subsubsection{Template Overrides\label{sec:tmploverrides}}

It is possible to override the templates located in a renderer's directory
with templates defined elsewhere.  This is done using the 
\environment{*TEMPLATES} environment variable.  The ``*'' in the name
\environment{*TEMPLATES} is a wildcard and must be replaced by the name of the
renderer.  For example, if you are using the XHTML renderer, the 
environment variable would be \environment{XHTMLTEMPLATES}.  For the PageTemplate
renderer, the environment variable would be \environment{PAGETEMPLATETEMPLATES}.

The format of this variable is the same as that of the \environment{PATH}
environment variable which means that you can put multiple directory 
names in this variable.  In addition, the environment variables for 
each of the parent renderers is also used, so that you can use
multiple layers of template directories.

You can actually create an entire renderer just using overrides and the
PT renderer.  Since the PT renderer doesn't actually define any templates,
it is just a framework for defining other XML/HTML renderers, you can
simply load the PT renderer and set the \environment{PAGETEMPLATETEMPLATES} 
environment
variable to the locations of your templates.  This method of creating 
renderers will work for any XML/HTML that doesn't require any special
post-processing.


\subsection{Defining and Using Themes}

In addition to the templates that define how each node should be rendered,
there are also templates that define page layouts.  Page layouts are used
whenever a node in the document generates a new file.   Page layouts 
generally include all of the markup required to make a complete document
of the desired DTD, and may include things like navigation buttons,
tables of contents, breadcrumb trails, etc. to link the current file to
other files in the document. 

When rendering files, the content of the
node is generated first, then that content is wrapped in a page layout.
The page layouts are defined the same way as regular templates; however,
they all include ``-layout'' at the end of the template name.  For 
example the sectioning commands in \LaTeX\ would use the layout templates
``section-layout'', ``subsection-layout'', ``subsubsection-layout'', etc.
Again, these templates can exist in files by themselves or multiply 
specified in a zpts file.  If no layout template exists for a particular
node, the template name ``default-layout'' is used.

Since there can be several themes defined within a renderer, theme files
are stored in a subdirectory of a renderer directory.  This directory
is named \file{Themes}.  The \file{Themes} directory itself only contains
directories that correspond to the themes themselves where the name
of the directory corresponds to the name of the theme.  These theme 
directories generally only consist of the layout files described above,
but can override other templates as well.  Below is a file listing
demonstrating the structure of a renderer with multiple themese.
\begin{verbatim}
# Renderer directory: contains template files
XHTML/

# Theme directory: contains theme directories
XHTML/Themes/

# Theme directories: contain page layout templates
XHTML/Themes/default/
XHTML/Themes/fancy/
XHTML/Themes/plain/
\end{verbatim}
\note{If no theme is specified in the document configuration, a theme
    with the name ``default'' is used.}

Since all template directories are created equally, you can also define
themes in template directories specified by environment variables as
described in section \ref{sec:tmploverrides}.  Also, theme files are 
searched in the same way as regular templates, so any theme defined
in a renderer superclass' directory is valid as well.


\subsection{Zope Page Template Tutorial\label{sec:zpttutorial}}

The Zope Page Template (ZPT) language is actually just a set of XML
attributes that can be applied to markup of an DTD.  These attributes
tell the ZPT interpreter how to process the element.  There are 
seven different attributes that you can use to direct the processing
of an XML or HTML file (in order of evaluation): define, condition, repeat, 
content, replace, attributes, and omit-tag.  These attributes are
described in section \ref{sec:talattributes}.  For a more complete description,
see the official ZPT documentation at 
\url{http://www.zope.org/Documentation/Books/ZopeBook/2_6Edition/ZPT.stx}.


\subsubsection{Template Attribute Language Expression Syntax (TALES)}

The Template Attribute Language Expression Syntax (TALES) is used
by the attribute language described in the next section.  The TALES 
syntax is used to evaluate expressions based on objects in the
template namespace.  The results of these expressions can be used to
define variables, produce output, or be used as booleans.  There are
also several operators used to modify the behavior or interpretation
of an expression.  The expressions and their modifiers are described
below.

\paragraph{path: operator\label{sec:pathoperator}}

A ``path'' is the most basic form on an expression in ZPT.  The basic form
is shown below.
\begin{verbatim}
[path:]string [ | TALES expression ]
\end{verbatim}

The \var{path:} operator is actually optional on all paths.  Leaving it
off makes no difference.  The ``string'' in the above syntax is a '/'
delimited string of names.  Each name refers to a property of the 
previous name in the string.  Properties can include attributes, methods,
or keys in a dictionary.  These properties can in turn have properties
of their own.  Some examples of paths are shown below.
\begin{verbatim}
# Access the parentNode attribute of chapter, then get its title
chapter/parentNode/title

# Get the key named 'foo' from the dictionary bar
bar/foo

# Call the title method on the string in the variable booktitle
booktitle/title
\end{verbatim}

It is possible to specify multiple paths separated by a pipe (|).
These paths are evaluated from left to right.  The first one to return
a non-None value is used.
\begin{verbatim}
# Look for the title on the current chapter node as well as its parents
chapter/title | chapter/parentNode/title | chapter/parentNode/parentNode/title

# Look for the value of the option otherwise get its default value
myoptions/encoding | myoptions/defaultencoding
\end{verbatim}

There are a few keywords that can be used in place of a path in a
TALES expression as well.
\begin{tableii}{l|p{4in}}{var}{Name}{Purpose}
\lineii{nothing}{same as \var{None} in Python}
\lineii{default}{keeps whatever the existing value of the element or attribute is}
\lineii{options}{dictionary of values passed in to the template when instatiated}
\lineii{repeat}{the repeat variable (see \ref{sec:talrepeat})}
\lineii{attrs}{dictonary of the original attributes of the element}
\lineii{CONTEXTS}{dictionary containing all of the above}
\end{tableii}


\paragraph{exists: operator}

This operator returns true if the path exists.  If the path does not exist,
the operator returns false.   The syntax is as follows.
\begin{verbatim}
exists:path
\end{verbatim}

The ``path'' in the code above is a path as described in section 
\ref{sec:pathoperator}.  This operator is commonly combined with the
not: operator.


\paragraph{nocall: operator}

By default, if a property that is retrieved is callable, it will be
called automatically.  Using the nocall: operator, prevents this 
execution from happening.  The syntax is shown below.
\begin{verbatim}
nocall:path
\end{verbatim}


\paragraph{not: operator}

The not: operator simply negates the boolean result of the path.  If
the path is a boolean true, the not: operator will return false, and
vice versa.  The syntax is shown below.
\begin{verbatim}
not:path
\end{verbatim}


\paragraph{string: operator}

The string: operator allows you to combine literal strings and paths
into one string.  Paths are inserted into the literal string using a
syntax much like that of Python Templates: \$path or \$\{path\}.
The general syntax is:
\begin{verbatim}
string:text
\end{verbatim}

Here are some examples of using the string: operator.
\begin{verbatim}
string:Next - ${section/links/next}
string:($pagenumber)
string:[${figure/number}] ${figure/caption}
\end{verbatim}


\paragraph{python: operator}

The python: operator allows you to evaluate a Python expression.  The
syntax is as follows.
\begin{verbatim}
python:python-code
\end{verbatim}

The ``python-code'' in the expression above can include any of the Python
built-in functions and operators as well as four new functions that
correspond to the TALES operators: path, string, exists, and nocall.
Each of these functions takes a string containing the path to be 
evaluated (e.g. path('foo/bar'), exists('chapter/title'), etc.).

When using Python operators, you must escape any characters that would
not be legal in an XML/HTML document (i.e. <>\&).  For example, 
to write an expression to test if a number was less than or greater than
two numbers, you would need to do something like the following example.
\begin{verbatim}
# See if the figure number is less than 2 or greater than 4
python: path('figure/number') &lt; 2 or path('figure/number') &gt; 4
\end{verbatim}


\paragraph{stripped: operator}

The stripped: operator only exists in the SimpleTAL distribution provided
by \plasTeX.  It evaluates the given path and removes any markup from
that path.  Essentially, it is a way to get a plain text representation
of the path.  The syntax is as follows.
\begin{verbatim}
stripped:path
\end{verbatim}


\subsubsection{Template Attribute Language (TAL) Attributes\label{sec:talattributes}}

\paragraph{tal:define}

The \attr{tal:define} attribute allows you to define a variable for use
later in the template.  Variables can be specifies as local (only for
use in the scope of the current element) or global (for use anywhere in
the template).  The syntax of the define attribute is shown below.
\begin{verbatim}
tal:define="[ local | global ] name expression [; define-expression ]"
\end{verbatim}

The define attributes sets the value of ``name'' to ``expression.''  
By default, the scope of the variable is local, but can be specified
as global by including the ``global'' keyword before the name of the 
variable.  As shown in the grammar above, you can specify multiple 
variables in one \attr{tal:define} attribute by separating the define
expressions by semi-colons.

Examples of using the \attr{tal:define} attribute are shown belaw.
\begin{verbatim}
<p tal:define="global title document/title; 
               next self/links/next;
               previous self/links/previous;
               length python:len(self);
               up string:Up - ${self/links/up}">
...
</p>
\end{verbatim}


\paragraph{tal:condition}

The \attr{tal:condition} attribute allows you to conditionally include
an element.  The syntax is shown below.
\begin{verbatim}
tal:condition="expression"
\end{verbatim}

The \attr{tal:condition} attribute is very simple.  If the expression
evaluates to true, the element and its children will be evaluated and 
included in the output.  If the expression evaluates to false, the element 
and its children will not be evaluated or included in the output.
Valid expressions for the \attr{tal:condition} attribute are the same 
as those for the expressions in the \attr{tal:define} attribute.
\begin{verbatim}
<p tal:condition="python:len(self)">
    <b tal:condition="self/caption">Caption for paragraph</b>
    ...
</p>
\end{verbatim}


\paragraph{tal:repeat\label{sec:talrepeat}}

The \attr{tal:repeat} attribute allows you to repeat an element multiple 
times; the syntax is shown below.
\begin{verbatim}
tal:repeat="name expression"
\end{verbatim}

When the \attr{tal:repeat} attribute is used on an element, the
result of``expression'' is iterated over, and a new element is generated
for each item in the iteration.  The value of the current item is
set to ``name'' much like in the \attr{tal:define} attribute.

Within the scope of the repeated element, another variable is available:
\var{repeat}.  This variable contains several properties related to
the loop.  
\begin{tableii}{l|p{4in}}{var}{Name}{Purpose}
\lineii{index}{number of the current iteration starting from zero}
\lineii{number}{number of the current iteration starting from one}
\lineii{even}{is true if the iteration number is even}
\lineii{odd}{is true if the iteration number is odd}
\lineii{start}{is true if this is the first iteration}
\lineii{end}{is true if this is the last iteration; This is never
    true if the repeat expression returns an iterator}
\lineii{length}{the length of the sequence being iterated over; This
    is set to \var{sys.maxint} for iterators.}
\lineii{letter}{lower case letter corresponding to the current iteration
    number starting with 'a'}
\lineii{Letter}{upper case letter corresponding to the current iteration
    number starting with 'A'}
\lineii{roman}{lower case Roman numeral corresponding to the current iteration
    number starting with 'i'}
\lineii{Roman}{upper case Roman numeral corresponding to the current iteration
    number starting with 'I'}
\end{tableii}

To access the properties listed above, you must use the property of 
the \var{repeat} variable that corresponds to the repeat variable name.
For example, if your repeat variable name is ``item'', you would access
the above variables using the expressions \var{repeat/item/index}, 
\var{repeat/item/number}, \var{repeat/item/even}, etc.

A simple example of the \attr{tal:repeat} attribute is shown below.
\begin{verbatim}
<ol>
<li tal:repeat="option options" tal:content="option/name">option name</li>
</ol>
\end{verbatim}

One commonly used feature of rendering tables is alternating row colors.
This is a little bit tricky with ZPT since the \attr{tal:condition}
attribute is evaluated before the \attr{tal:repeat} directive.  You
can get around this by using the \namespace{metal} namespace.  This
is the namespace used by ZPT's macro language\footnote{The macro language
isn't discussed here.  See the official ZPT documentation for more 
information.}  You can create another element around the element you 
want to be conditional.  This wrapper element is simply there to do the 
iterating, but is not included in the output.  The example below shows
how to do alternating row colors in an HTML table.
\begin{verbatim}
<table>
<metal:block tal:repeat="employee employees">
<!-- even rows -->
<tr tal:condition="repeat/employee/even" style="background-color: white">
    <td tal:content="employee/name"></td>
    <td tal:content="employee/title"></td>
</tr>
<!-- odd rows -->
<tr tal:condition="repeat/employee/odd" style="background-color: gray">
    <td tal:content="employee/name"></td>
    <td tal:content="employee/title"></td>
</tr>
</metal:block>
</table>
\end{verbatim}


\paragraph{tal:content}

The \attr{tal:content} attribute evaluates an expression and replaces
the content of the element with the result of the expression.  The
syntax is shown below.
\begin{verbatim}
tal:content="[ text | structure ] expression"
\end{verbatim}

The \var{text} and \var{structure} options in the \attr{tal:content}
attribute indicate whether or not the content returned by the 
expression should be escaped (i.e. "\&<> replaced by \&quot;, \&amp;, \&lt;,
and \&gt;, respectively).  When the \var{text} option is used, these
special characters are escaped; this is the default behavior.  When
the \var{structure} option is specified, the result of the expression is
assumed to be valid markup and is not escaped.  

In SimpleTAL, the default
behavior is the same as using the \var{text} option.  However, in 
\plasTeX, 99.9\% of the time the content returned by the expression is
valid markup, so the default was changed to \var{structure} in the 
SimpleTAL package distributed with \plasTeX.


\paragraph{tal:replace}

The \attr{tal:replace} attribute is much like the \attr{tal:content}
attribute.  They both evaluate an expression and include the content
of that expression in the output, and they both have a \var{text} and
\var{structure} option to indicate escaping of special characters.
The difference is that when the \attr{tal:replace} attribute is used,
the element with the \attr{tal:replace} attribute on it is not included
in the output.  Only the content of the evaluated expression is returned.
The syntax of the \attr{tal:replace} attribute is shown below.
\begin{verbatim}
tal:replace="[ text | structure ] expression"
\end{verbatim}


\paragraph{tal:attributes}

The \attr{tal:attributes} attribute allows you to programatically create
attributes on the element.  The syntax is shown below.
\begin{verbatim}
tal:attributes="name expression [; attribute-expression ]"
\end{verbatim}

The syntax of the \attr{tal:attributes} attribute is very similar to
that of the \attr{tal:define} attribute.  However, in the case of the
\attr{tal:attributes} attribute, the name is the name of the attribute
to be created on the element and the expression is evaluated to
get the value of the attribute.  If an error occurs or \var{None} is 
returned by the expression, then the attribute is removed from the 
element.

Just as in the case of the \attr{tal:define} attribute, you can specify
multiple attributes separated by semi-colons (;).  If a semi-colon character
is needed in the expression, then it must be represented by a double
semi-colon (;;).

An example of using the \attr{tal:attributes} is shown below.
\begin{verbatim}
<a tal:attributes="href self/links/next/url; 
                   title self/links/next/title">link text</a>
\end{verbatim}


\paragraph{tal:omit-tag}

The \attr{tal:omit-tag} attribute allows you to conditionally omit an
element.  The syntax is shown below.
\begin{verbatim}
tal:omit-tag="expression"
\end{verbatim}

If the value of ``expression'' evaluates to true (or is empty), the element 
is omitted; however, the content of the element is still sent to the output.
If the expression evaluates to false, the element is included in the 
output.


\section{XHTML Renderer}

The XHTML renderer is a subclass of the ZPT renderer (section \ref{sec:zpt}).
Since the ZPT renderer can render any variant of XML or HTML, the 
XHTML renderer has very little to do in the Python code.  Almost all
of the additionaly processing in the XHTML renderer has to do with
generated images.  Since HTML cannot render \LaTeX's vector graphics
or equations natively, they are converted to images.  In order for 
inline equations to line up correctly with the text around them, CSS
attributes are used to adjust the vertical alignment.  Since the images
aren't generated until after all of the document has been rendered,
this CSS information is added in post-processing (i.e. the \method{cleanup} 
method).

In addition to the processing of images, all characters with a ordinal
greater than 127 are converted into numerical entities.  This should 
prevent any rendering problems due to unknown encodings.

Most of the work in this renderer was in creating the templates for 
every \LaTeX\ construct.  Since this renderer was intended to be the
basis of all HTML-based renderers, it must be capable of rendering
all \LaTeX\ constructs; therefore, there are ZPT templates for every
\LaTeX\ command, and the commands in some common \LaTeX\ packages.

While the XHTML renderer is fairly complete when it comes to standard
\LaTeX, there are many packages which are not currently supported.
To add support for these packages, templates (and possibly Python
based macros; section \ref{sec:macros}) must be created.


\subsection{Themes}

The theming support in the XHTML renderer is the same as that of the
ZPT renderer.  Any template directory can have a subdirectory called
\file{Themes} which contains theme directories with sets of templates
in them.  The names of the directories in the \file{Themes} directory
corresponds to the name of the theme.  There are currently two themes
included with \plasTeX: default and plain.  The default theme is a
minor variation of the one used in the Python 1.6 documentation.  The
plain theme is a theme with no extra navigation bars. 


\section{tBook Renderer}

Not yet implemented.

\section{DocBook Renderer}

Not yet implemented.



\chapter{\plasTeX\ Frameworks and APIs}

\input{macros-api}

\section{\module{plasTeX.ConfigManager} --- \plasTeX\ Configuration}
\label{sec:configuration-api}

\declaremodule{standard}{plasTeX.ConfigManager}
\modulesynopsis{\plasTeX's configuration management system}

The configuration system in \plasTeX\ that parses the command-line options
and configuration files is very flexible.  While many options are setup
by the \plasTeX\ framework, it is possible for you to add your own options.
This is useful if you have macros that may need to be configured by 
configurable options, or if you write a renderer that surfaces special options
to control it.

The config files that \class{ConfigManager} supports are standard INI-style
files.  This is the same format supported by Python's \class{ConfigParser}.
However, this API has been extended with some dictionary-like behaviors
to make it more Python friendly.

In addition to the config files, \class{ConfigManager} can also parse
command-line options and merge the options from the command-line into
the options set by the given config files.  In fact, when adding options
to a \class{ConfigManager}, you specify both how they appear in the config
file as well as how they appear on the command-line.  Below is a basic 
example.

\begin{verbatim}
from plasTeX.ConfigManager import *
c = ConfigManager()

# Create a new section in the config file.  This corresponds to the
# [ sectionname ] sections in an INI file.  The returned value is 
# a reference to the new section
d = c.add_section('debugging')

# Add an option to the 'debugging' section called 'verbose'.
# This corresponds to the config file setting:
#
# [debugging]
# verbose = no
#
d['verbose'] = BooleanOption(
    """ Increase level of debugging information """,
    options = '-v --verbose !-q !--quiet',
    default = False,
)

# Read system-level config file
c.read('/etc/myconfig.ini')

# Read user-level config file
c.read('~/myconfig.ini')

# Parse the current command-line arguments
opts, args = c.getopt(sys.argv[1:])

# Print the value of the 'verbose' option in the 'debugging' section
print c['debugging']['verbose']
\end{verbatim}

One interesting thing to note about retrieving values from a 
\class{ConfigManager} is that you get the value of the option
rather than the option instance that you put in.  For example, in the
code above.  A \class{BooleanOption} in put into the `verbose' option
slot, but when it is retrieved in the \function{print} statement at
the end, it prints out a boolean value.  This is true of all option
types.  You can access the option instance in the \member{data} attribute
of the section (e.g. \code{c['debugging'].data['verbose']}).


\subsection{ConfigManager Objects}

\begin{classdesc}{ConfigManager}{defaults=\{~\}}
Instantiate a configuration class for \plasTeX\ that parses the command-line options
as well as reads the config files.  

The optional argument, \var{defaults},
is a dictionary of default values for the configuration object.  These 
values are used if a value is not found in the requested section.
\end{classdesc}

\begin{methoddesc}[ConfigManager]{__add__}{other}
merge items from another \class{ConfigManager}.  This allows you to add
\class{ConfigManager} instances with syntax like: config + other. 
This operation will modify the original instance.
\end{methoddesc}

\begin{methoddesc}[ConfigManager]{add_section}{name}
create a new section in the configuration with the given name.  This 
name is the name used for the section heading in the INI file (i.e. the
name used within square brackets (\lbrack~\rbrack) to start a section).
The return value of this method is a reference to the newly created section.  
\end{methoddesc}

\begin{methoddesc}[ConfigManager]{categories}{}
return the dictionary of categories
\end{methoddesc}

\begin{methoddesc}[ConfigManager]{copy}{}
return a deep copy of the configuration
\end{methoddesc}

\begin{methoddesc}[ConfigManager]{defaults}{}
return the dictionary of default values
\end{methoddesc}

\begin{methoddesc}[ConfigManager]{read}{filenames}
read configuration data contained in files specified by \var{filenames}.
Files that cannot be opened are silently ignored.  This is designed so that
you can specify a list of potential configuration file locations (e.g.
current directory, user's home directory, system directory), and all 
existing configuration files in the list will be read.  A single filename
may also be given.
\end{methoddesc}

\begin{methoddesc}[ConfigManager]{get}{section, option, raw=0, vars=\{~\}}
retrieve the value of \var{option} from the section \var{section}.
Setting \var{raw} to true prevents any string interpolation from occurring
in that value.  \var{vars} is a dictionary of addition value to use 
when interpolating values into the option.

\note{You can alsouse the alternative dictionary syntax: config[section].get(option).}
\end{methoddesc}

\begin{methoddesc}[ConfigManager]{getboolean}{section, option}
retrieve the specified value and cast it to a boolean
\end{methoddesc}

\begin{methoddesc}[ConfigManager]{get_category}{key}
return the title of the given category
\end{methoddesc}

\begin{methoddesc}[ConfigManager]{getfloat}{section, option}
retrieve the specified value and cast it to a float
\end{methoddesc}

\begin{methoddesc}[ConfigManager]{getint}{section, option}
retrieve the specified value and cast it to and integer
\end{methoddesc}

\begin{methoddesc}[ConfigManager]{get_opt}{section, option}
return the option value with any leading and trailing quotes removed
\end{methoddesc}

\begin{methoddesc}[ConfigManager]{getopt}{args=None, merge=True}
parse the command-line options.  If \var{args} is not given, the 
args are parsed from \code{sys.argv[1:]}.  If \var{merge} is set to false,
then the options are not merged into the configuration.  The return value
is a two element tuple.  The first value is a list of parsed options
in the form \code{(option, value)}, and the second value is the list of
arguments.
\end{methoddesc}

\begin{methoddesc}[ConfigManager]{get_optlist}{section, option, delim=','}
return the option value as a list using \var{delim} as the delimiter
\end{methoddesc}

\begin{methoddesc}[ConfigManager]{getraw}{section, option}
return the raw (i.e. un-interpolated) value of the option
\end{methoddesc}

\begin{methoddesc}[ConfigManager]{has_category}{key, title}
add a category to group options when printing the command-line help.
Command-line options can be grouped into categories to make options easier
to find when printing the usage message for a program.  Categories consist
of two pieces: 1) the name, and 2) the title.  The name is the key in
the category dictionary and is the name used when specifying which category
an option belongs to.  The title is the actual text that you see as a 
section header when printing the usage message.
\end{methoddesc}

\begin{methoddesc}[ConfigManager]{has_option}{section, name}
return a boolean indicating whether or not an option with the given name exists
in the given section
\end{methoddesc}

\begin{methoddesc}[ConfigManager]{has_section}{name}
return a boolean indicating whether or not a section with the given name exists
\end{methoddesc}

\begin{methoddesc}[ConfigManager]{__iadd__}{other}
merge items from another \class{ConfigManager}.  This allows you to add
\class{ConfigManager} instances with syntax like: config += other.
\end{methoddesc}

\begin{methoddesc}[ConfigManager]{options}{name}
return a list of configured option names within a section.  Options are all
of the settings of a configuration file within a section (i.e. the lines that 
start with `optionname=').
\end{methoddesc}

\begin{methoddesc}[ConfigManager]{__radd__}{other}
merge items from another \class{ConfigManager}.  This allows you to add
\class{ConfigManager} instances with syntax like: other + config.
This operation will modify the original instance.
\end{methoddesc}

\begin{methoddesc}[ConfigManager]{readfp}{fp, filename=None}
like \method{read()}, but the argument is a file object.  The optional
\var{filename} argument is used for printing error messages.
\end{methoddesc}

\begin{methoddesc}[ConfigManager]{remove_option}{section, option}
remove the specified option from the given section
\end{methoddesc}

\begin{methoddesc}[ConfigManager]{remove_section}{section}
remove the specified section
\end{methoddesc}

\begin{methoddesc}[ConfigManager]{__repr__}{}
return the configuration as an INI formatted string; this also includes 
options that were set from Python code.
\end{methoddesc}

\begin{methoddesc}[ConfigManager]{sections}{}
return a list of all section names in the configuration
\end{methoddesc}

\begin{methoddesc}[ConfigManager]{set}{section, option, value}
set the value of an option
\end{methoddesc}

\begin{methoddesc}[ConfigManager]{__str__}{}
return the configuration as an INI formatted string; however, do not 
include options that were set from Python code.
\end{methoddesc}

\begin{methoddesc}[ConfigManager]{to_string}{source=...}
return the configuration as an INI formatted string.  The \var{source}
option indicates which source of information should be included in
the resulting INI file.  The possible values are:
\begin{tableii}{l|l}{var}{Name}{Description}
\lineii{COMMANDLINE}{set from a command-line option}
\lineii{CONFIGFILE}{set from a configuration file}
\lineii{BUILTIN}{set from Python code}
\lineii{ENVIRONMENT}{set from an environment variable}
\end{tableii}
\end{methoddesc}

\begin{methoddesc}[ConfigManager]{write}{fp}
write the configuration as an INI formatted string to the given file object
\end{methoddesc}

\begin{methoddesc}[ConfigManager]{usage}{categories=[]}
print the descriptions of all command-line options.  If \var{categories}
is specified, only the command-line options from those categories is printed.
\end{methoddesc}


\subsection{ConfigSection Objects}

\begin{classdesc}{ConfigSection}{name, data=\{~\}}
Instantiate a \class{ConfigSection} object.  

\var{name} is the name of the section.

\var{data}, if specified, is the dictionary of data to initalize
the section contents with.
\end{classdesc}

\class{ConfigSection} objects are rarely instantiated manually.  They 
are generally created using the \class{ConfigManager} API (either the 
direct methods or the Python dictionary syntax).

\begin{memberdesc}[ConfigSection]{data}
dictionary that contains the option instances.  This is only accessed
if you want to retrieve the real option instances.  Normally, you would
use standard dictionary key access syntax on the section itself to
retrieve the option values.
\end{memberdesc}

\begin{memberdesc}[ConfigSection]{name}
the name given to the section.
\end{memberdesc}


\begin{methoddesc}[ConfigSection]{copy}{}
make a deep copy of the section object.
\end{methoddesc}

\begin{methoddesc}[ConfigSection]{defaults}{}
return the dictionary of default options associated with the parent
\class{ConfigManager}.
\end{methoddesc}

\begin{methoddesc}[ConfigManager]{get}{option, raw=0, vars=\{~\}}
retrieve the value of \var{option}.
Setting \var{raw} to true prevents any string interpolation from occurring
in that value.  \var{vars} is a dictionary of addition value to use 
when interpolating values into the option.

\note{You can alsouse the alternative dictionary syntax: section.get(option).}
\end{methoddesc}

\begin{methoddesc}[ConfigManager]{getboolean}{section, option}
retrieve the specified value and cast it to a boolean
\end{methoddesc}

\begin{methoddesc}[ConfigSection]{__getitem__}{key}
retrieve the value of an option.  This method allows you to use Python's 
dictionary syntax on a section as shown below.
\begin{verbatim}
# Print the value of the 'optionname' option
print mysection['optionname'] 
\end{verbatim}
\end{methoddesc}

\begin{methoddesc}[ConfigManager]{getint}{section, option}
retrieve the specified value and cast it to and integer
\end{methoddesc}

\begin{methoddesc}[ConfigManager]{getfloat}{section, option}
retrieve the specified value and cast it to a float
\end{methoddesc}

\begin{methoddesc}[ConfigManager]{getraw}{section, option}
return the raw (i.e. un-interpolated) value of the option
\end{methoddesc}

\begin{memberdesc}[ConfigSection]{parent}
a reference to the parent \class{ConfigManager} object.
\end{memberdesc}

\begin{methoddesc}[ConfigSection]{__repr__}{}
return a string containing an INI file representation of the section.
\end{methoddesc}

\begin{methoddesc}[ConfigSection]{set}{option, value}
create a new option or set an existing option with the name 
\var{option} and the value of \var{value}.
If the given value is already an option instance, it is simply inserted
into the section.  If it is not an option instance, an appropriate type
of option is chosen for the given type.
\end{methoddesc}

\begin{methoddesc}[ConfigSection]{__setitem__}{key, value}
create a new option or set an existing option with the name \var{key} and
the value of \var{value}.  This method allows you to use Python's 
dictionary syntax to set options as shown below.
\begin{verbatim}
# Create a new option called 'optionname'
mysection['optionname'] = 10
\end{verbatim}
\end{methoddesc}

\begin{methoddesc}[ConfigSection]{__str__}{}
return a string containing an INI file representation of the section.
Options set from Python code are not included in this representation.
\end{methoddesc}

\begin{methoddesc}[ConfigSection]{to_string}{\optional{source}}
return a string containing an INI file representation of the section.
The \var{source} option allows you to only display options from 
certain sources.  See the \method{ConfigManager.source()} method for more
information.
\end{methoddesc}


\subsection{Configuration Option Types}

There are several option types that should cover just about any type 
of command-line and configuration option that you may have.  However,
in the spirit of object-orientedness, you can, of course, subclass
one of these and create your own types.  \class{GenericOption} is the 
base class for all options.  It contains all of the underlying framework
for options, but should never be instantiated directly.  Only subclasses
should be instantiated.

\begin{classdesc}{GenericOption}{
    \optional{docsTring, options, default, optional, values, category,
              callback, synopsis, environ, registry, mandatory,
              name, source}
}
Declare a command line option.

Instances of subclasses of \class{GenericOption} must be placed in
a \class{ConfigManager} instance to be used.  See the documentation for
\class{ConfigManager} for more details.

\var{docstring} is a string in the format of Python documentation
   strings that describes the option and its usage.  The first
   line is assumed to be a one-line summary for the option.
   The following paragraphs are assumed to be a complete description
   of the option.  You can give a paragraph with the label
   'Valid Values:' that contains a short description of the
   values that are valid for the current option.  If this
   paragraph exists and an error is encountered while validating
   the option, this paragraph will be printed instead of the
   somewhat generic error message for that option type.

\var{options} is a string containing all possible variants of the
   option.  All variants should contain the '-', '--', etc. at
   the beginning.  For boolean options, the option can be preceded
   by a '!' to mean that the option should be turned OFF rather
   than ON which is the default.

\var{default} is a value for the option to take if it isn't specified
   on the command line

\var{optional} is a value for the option if it is given without a value.
   This is only used for options that normally take a value,
   but you also want a default that indicates that the option
   was given without a value.

\var{values} defines valid values for the option.  This argument can take
   the following forms:
        

\begin{tableii}{l|p{4in}}{var}{Type}{Description}
   \lineii{single value}{for \class{StringOption} this this is a string, for
      \class{IntegerOption} this is an integer, for \class{FloatOption} this is
      a float.  The single value mode is most useful when the 
      value is a regular expression.  For example, to specify
      that a \class{StringOption} must be a string of characters followed
      by a digit, 'values' would be set to \code{re.compile(r'{\textbackslash}w+{\textbackslash}d')}.}
   \lineii{range of values}{a two element list can be given to specify
      the endpoints of a range of valid values.  This is probably
      most useful on \class{IntegerOption} and \class{FloatOption}.  
      For example, to specify that an IntegerOption can only take the values
      from 0 to 10, 'values' would be set to [0,10]. 
      \note{This mode must \emph{always} use a Python list since using
            a tuple means something else entirely.}}
   \lineii{tuple of values}{a tuple of values can be used to specify 
      a complete list of valid values.  For example, to specify
      that an \class{IntegerOption} can take the values 1, 2, or 3, 'values'
      would be set to \code{(1,2,3)}.  If a string value can only take 
      the values, 'hi', 'bye', and any string of characters beginning
      with the letter 'z', 'values' would be set to 
      \code{('hi','bye',re.compile(r'z.*?'))}.
      \note{This mode must *always* use a Python tuple since using
            a list means something else entirely.}}
\end{tableii}

\var{category} is a category key which specifies which category the
   option belongs to (see the \class{ConfigManager} documentation on 
   how to create categories).

\var{callback} is a function to call after the value of the option has
   been validated.  This function will be called with the validated
   option value as its only argument.

\var{environ} is an environment variable to use as default value instead
   of specified value.  If the environment variable exists, it
   will be used for the default value instead of the specified value.

\var{registry} is a registry key to use as default value instead of
   specified value.  If the registry key exists, it will be used
   for the default value instead of the specified value.  A
   specified environment variable takes precedence over this value.
   \note{This is not implemented yet.}

\var{name} is a key used to get the option from its corresponding section.
   You do not need to specify this.  It will be set automatically
   when you put the option into the \class{ConfigManager} instance.

\var{mandatory} is a flag used to determine if the option itself is
   required to be present.  The idea of a "mandatory option" is
   a little strange, but I have seen it done.

\var{source} is a flag used to determine whether the option was set
   directly in the \class{ConfigManager} instance through Python,
   by a configuration file/command line option, etc.  You do not need
   to specify this, it will be set automatically during parsing.
   This flag should have the value of \var{BUILTIN}, \var{COMMANDLINE},
   \var{CONFIGFILE}, \var{ENVIRONMENT}, \var{REGISTRY}, or \var{CODE}.
\end{classdesc}

\begin{methoddesc}[GenericOption]{acceptsArgument}{}
return a boolean indicating whether or not the option accepts an argument on 
the command-line.  For example, boolean options do not accept an argument.
\end{methoddesc}

\begin{methoddesc}[GenericOption]{cast}{arg}
cast the given value to the appropriate type.
\end{methoddesc}

\begin{methoddesc}[GenericOption]{checkValues}{value}
check \var{value} against all possible valid values for the option.
If the value is invalid, raise an \exception{InvalidOptionError} exception.
\end{methoddesc}

\begin{methoddesc}[GenericOption]{clearValue}{}
reset the value of the option as if it had never been set.
\end{methoddesc}

\begin{methoddesc}[GenericOption]{getValue}{\optional{default}}
return the current value of the option.  If \var{default} is specified
and a value cannot be gotten from any source, it is returned.
\end{methoddesc}


\begin{methoddesc}[GenericOption]{__repr__}{}
return a string containing a command-line representation of the option and
its value.
\end{methoddesc}

\begin{methoddesc}[GenericOption]{requiresArgument}{}
return a boolean indicating whether or not the option requires an argument
on the command-line.
\end{methoddesc}

As mentioned previously, \class{GenericOption} is an abstract class 
(i.e. it should not be instantiated directly).  Only subclasses of 
\class{GenericOption} should be instantiated.  Below are some examples
of use of some of these subclasses, followed by the descriptions
of the subclasses themselves.

\begin{verbatim}
   BooleanOption(
      ''' Display help message ''',
      options = '--help -h',
      callback = usage,  # usage() function must exist prior to this
   )
\end{verbatim}

\begin{verbatim}
   BooleanOption(
      ''' Set verbosity ''',
      options = '-v --verbose !-q !--quiet',
   )
\end{verbatim}

\begin{verbatim}
   StringOption(
      '''
      IP address option

      This option accepts an IP address to connect to.

      Valid Values:
      '#.#.#.#' where # is a number from 1 to 255

      ''',
      options = '--ip-address',
      values = re.compile(r'\d{1,3}(\.\d{1,3}){3}'),
      default = '127.0.0.0',
      synopsis = '#.#.#.#',
      category = 'network',  # Assumes 'network' category exists
   )
\end{verbatim}

\begin{verbatim}
   IntegerOption(
      '''
      Number of seconds to wait before timing out

      Valid Values:
      positive integer

      ''',
      options = '--timeout -t',
      default = 300,
      values = [0,1e9],
      category = 'network',
   )
\end{verbatim}

\begin{verbatim}
   IntegerOption(
      '''
      Number of tries to connect to the host before giving up

      Valid Values:
      accepts 1, 2, or 3 retries

      ''',
      options = '--tries',
      default = 1,
      values = (1,2,3),
      category = 'network',
   )
\end{verbatim}

\begin{verbatim}
   StringOption(
      '''
      Nonsense option for example purposes only

      Valid Values:
      accepts 'hi', 'bye', or any string beginning with the letter 'z'

      ''',
      options = '--nonsense -n',
      default = 'hi',
      values = ('hi', 'bye', re.compile(r'z.*?')),
   )
\end{verbatim}

\begin{classdesc}{BooleanOption}{\optional{\class{GenericOption} arguments}}
Boolean options are simply options that allow you to specify an `on' or 
`off' state.  The accepted values for a boolean option in a config file
are `on', `off', `true', `false', `yes', `no', 0, and 1.  Boolean options on
the command-line do not take an argument; simply specifying the option
sets the state to true.

One interesting feature of boolean options is in specifying the command-line
options.  Since you cannot specify a value on the command-line (the existence
of the option indicates the state), there must be a way to set the state to
false.  This is done using the `not' operator (!).  When specifying the 
\var{options} argument of the constructor, if you prefix an command-line 
option with an exclamation point, the existence of that option indicates
a false state rather than a true state.  Below is an example of an \var{options}
value that has a way to turn debugging information on (\longprogramopt{debug}) 
or off (\longprogramopt{no-debug}).
\begin{verbatim}
BooleanOption( options = '--debug !--no-debug' )
\end{verbatim}
\end{classdesc}

\begin{classdesc}{CompoundOption}{\optional{\class{GenericOption} arguments}}
Compound options are options that contain multiple elements on the 
command-line.  They are simply groups of command-line arguments surrounded
by a pair of grouping characters (e.g. (~), [~], \{~\}, <~>).  This grouping
can contain anything including other command-line arguments.  However, 
all content between the grouping characters is unparsed.  This can be useful
if you have a program that wraps another program and you want to be able
to forward the wrapped program's options on.  An example of a compound option
used on the command-line is shown below.
\begin{verbatim}
# Capture the --diff-opts options to send to another program
mycommand --other-opt --diff-opts ( -ib --minimal ) file1 file2
\end{verbatim}
\end{classdesc}

\begin{classdesc}{CountedOption}{\optional{\class{GenericOption} arguments}}
A \class{CountedOption} is a boolean option that keeps track of how many
times it has been specified.  This is useful for options that control 
the verbosity of logging messages in a program where the number of times
an option is specified, the more logging information is printed.
\end{classdesc}

\begin{classdesc}{InputDirectoryOption}{\optional{\class{GenericOption} arguments}}
An \class{InputDirectoryOption} is an option that accepts a directory name
for input.
This directory name is checked to make sure that it exists and that it is
readable.  If it is not, a \exception{InvalidOptionError} exception is 
raised.
\end{classdesc}

\begin{classdesc}{OutputDirectoryOption}{\optional{\class{GenericOption} arguments}}
An \class{OutputDirectoryOption} is an option that accepts a directory name
for output.  If the directory exists, it is checked to make sure that it is
readable.  If it does not exist, it is created.
\end{classdesc}

\begin{classdesc}{InputFileOption}{\optional{\class{GenericOption} arguments}}
An \class{InputFileOption} is an option that accepts a file name for input.
The filename is checked to make sure that it exists and is readable.  If
it isn't, an \exception{InvalidOptionError} exception is raised.
\end{classdesc}

\begin{classdesc}{OutputFileOption}{\optional{\class{GenericOption} arguments}}
An \class{OutputFileOption} is an option that accepts a file name for output.
If the file exists, it is checked to make sure that it is writable.  
If a name contains a directory, the path is checked to make sure that it is 
writable.  If the directory does not exist, it is created. 
\end{classdesc}

\begin{classdesc}{FloatOption}{\optional{\class{GenericOption} arguments}}
A \class{FloatOption} is an option that accepts a floating point number.
\end{classdesc}

\begin{classdesc}{IntegerOption}{\optional{\class{GenericOption} arguments}}
An \class{IntegerOption} is an option that accepts an integer value.
\end{classdesc}

\begin{classdesc}{MultiOption}{\optional{\class{GenericOption} arguments, \optional{delim, range, template}}}
A \class{MultiOption} is an option that is intended to be used multiple times
on the command-line, or take a list of values.  Other options when specified
more than once simply overwrite the previous value.  \class{MultiOption}s
will append the new values to a list.

The delimiter used to separate multiple values is the comma (,).  A different
character can be specified in the \var{delim} argument.  

In addition, it is possible to specify the number of values that are legal
in the \var{range} argument.  The range argument takes a two element list.
The first element is the minimum number of times the argument is required.
The second element is the maximum number of times it is required.  You can
use a `*' (in quotes) to mean an infinite number.

You can cast each element in the list of values to a particular type by
using the \var{template} argument.  The \var{template} argument takes a
reference to the option class that you want the values to be converted to.
\end{classdesc}

\begin{classdesc}{StringOption}{\optional{\class{GenericOption} arguments}}
A \class{StringOption} is an option that accepts an arbitrary string.
\end{classdesc}

\input{dom-api}
\input{tex-api}

\section{\module{plasTeX.Context} --- The \TeX\ Context\label{sec:context-api}}

\declaremodule{standard}{plasTeX.Context}
\modulesynopsis{The \TeX\ document context}

The \class{Context} class stores all of the information associated with
the currently running document.  This includes things like macros, counters, 
labels, references, etc.  The context also makes sure that localized macros
get popped off when processing leaves a macro or environment.  The context
of a document also has the power to create new counters, dimens, if commands,
macros, as well as change token category codes.

Each time a \class{TeX} object is instantiated, it will create its own 
context.  This context will load all of the base macros and initialize
all of the context information described above.


\subsection{Context Objects}

\begin{classdesc}{Context}{\optional{load}}
Instantiate a new context.

If the \var{load} argument is set to true, the context will load all of the
base macros defined in \plasTeX.  This includes all of the macros used in 
the standard \TeX\ and \LaTeX\ distributions.
\end{classdesc}

\begin{memberdesc}[Context]{contexts}
stack of all macro and category code collections currently in the document 
being processed.  The item at index 0 include the global macro set and 
default category codes.
\end{memberdesc}

\begin{memberdesc}[Context]{counters}
a dictionary of counters.
\end{memberdesc}

\begin{memberdesc}[Context]{currentlabel}
the object that is given the label when a \macro{label} macro is invoked.
\end{memberdesc}

\begin{memberdesc}[Context]{isMathMode}
boolean that specifies if we are currently in \TeX's math mode or not.
\end{memberdesc}

\begin{memberdesc}[Context]{labels}
a dictionary of labels and the objects that they refer to.
\end{memberdesc}



\begin{methoddesc}[Context]{addGlobal}{key, value}
add a macro \var{value} with name \var{key} to the global namespace. 
\end{methoddesc}

\begin{methoddesc}[Context]{addLocal}{key, value}
add a macro \var{value} with name \var{key} to the current namespace. 
\end{methoddesc}

\begin{methoddesc}[Context]{append}{\optional{context}}
same as \method{push()}
\end{methoddesc}

\begin{methoddesc}[Context]{catcode}{char, code}
set the category code for a character in the current scope.  \var{char}
is the character that will have its category code changed.  \var{code}
is the \TeX\ category code (0-15) to change it to.
\end{methoddesc}

\begin{methoddesc}[Context]{chardef}{name, num}
create a new \TeX\ chardef like \macro{chardef}.

\var{name} is the name of the command to create.

\var{num} is the character number to use.
\end{methoddesc}

\begin{methoddesc}[Context]{__getitem__}{key}
look through the stack of macros and return the one with the name \var{key}.
The return value is an \emph{instance} of the requested macro,
not a reference to the macro class.
This method allows you to use Python's dictionary syntax to retrieve
the item from the context as shown below.
\begin{verbatim}
tex.context['section']
\end{verbatim}
\end{methoddesc}

\begin{methoddesc}[Context]{importMacros}{context}
import macros from another context into the global namespace.  The argument,
\var{context}, must be a dictionary of macros.
\end{methoddesc}

\begin{methoddesc}[Context]{label}{label}
set the given label to the currently labelable object.  An object can 
only have one label associated with it.
\end{methoddesc}

\begin{methoddesc}[Context]{let}{dest, source}
create a new \TeX\ let like \macro{let}.

\var{dest} is the command sequence to create.

\var{source} is the token to set the command sequence equivalent to.

\textbf{Example}
\begin{verbatim}
c.let('bgroup', BeginGroup('{'))
\end{verbatim}
\end{methoddesc}

\begin{methoddesc}[Context]{loadBaseMacros}{}
imports all of the base macros defined by \plasTeX.  This includes all of
the macros specified by the \TeX\ and \LaTeX\ systems.
\end{methoddesc}

\begin{methoddesc}[Context]{loadLanguage}{language, document}
loads a language package to configure names such as \macro{figurename},
\macro{tablename}, etc. See Section~\ref{sec:context-language} for more
information.

\var{language} is a string containing the name of the language file to load.

\var{document} is the document object being processed.
\end{methoddesc}

\begin{methoddesc}[Context]{loadINIPackage}{inifile}
load an INI formatted package file (see section \ref{sec:packages} for 
more information).
\end{methoddesc}

\begin{methoddesc}[Context]{loadPackage}{tex, file, \optional{options}}
loads a \LaTeX\ package.  

\var{tex} is the \TeX\ processor to use in parsing the package content

\var{file} is the name of the package to load

\var{options} is a dictionary containing the options to pass to the package.
This generally comes from the optional argument on a \macro{usepackage}
or \macro{documentclass} macro.

The package being loaded by this method can be one of three type: 1)
a native \LaTeX\ package, 2) a Python package, or 3) an INI formatted file.
The Python version of the package is searched for first.  If it is found,
it is loaded and an INI version of the package is also loaded if it
exists.  If there is no Python version, the true \LaTeX\ version of the
package is loaded.  If there is an INI version of the package in the same
directory as the \LaTeX\ version, that file is loaded also.
\end{methoddesc}

\begin{methoddesc}[Context]{newcommand}{name\optional{, nargs\optional{, definition\optional{, opt}}}}
create a new \LaTeX\ command like \macro{newcommand}.

\var{name} is the name of the macro to create.

\var{nargs} is the number of arguments including optional arguments.

\var{definition} is a string containing the macro definition.

\var{opt} is a string containing the default optional value.

\textbf{Examples}
\begin{verbatim}
c.newcommand('bold', 1, r'\\textbf{#1}')
c.newcommand('foo', 2, r'{\\bf #1#2}', opt='myprefix')
\end{verbatim}
\end{methoddesc}

\begin{methoddesc}[Context]{newcount}{name\optional{, initial}}
create a new count like \macro{newcount}.
\end{methoddesc}

\begin{methoddesc}[Context]{newcounter}{name, \optional{resetby, initial, format}}
create a new counter like \macro{newcounter}.

\var{name} is the name of the counter to create.

\var{resetby} is the counter that, when incremented, will reset the 
new counter.

\var{initial} is the initial value for the counter.

\var{format} is the printed format of the counter.

In addition to creating a new counter macro, another macro corresponding
to the \macro{the\var{name}} is created which prints the value of the
counter just like in \LaTeX.
\end{methoddesc}

\begin{methoddesc}[Context]{newdef}{name\optional{, args\optional{, definition\optional{, local}}}}
create a new \TeX\ definition like \macro{def}.

\var{name} is the name of the definition to create.

\var{args} is a string containing the \TeX\ argument profile.

\var{definition} is a string containing the macro code to expand when the
definition is invoked.

\var{local} is a boolean that specifies that the definition should only
exist in the local scope.  The default value is true.

\textbf{Examples}
\begin{verbatim}
c.newdef('bold', '#1', '{\\bf #1}')
c.newdef('put', '(#1,#2)#3', '\\dostuff{#1}{#2}{#3}')
\end{verbatim}
\end{methoddesc}

\begin{methoddesc}[Context]{newdimen}{name\optional{, initial}}
create a new dimen like \macro{newdimen}.
\end{methoddesc}

\begin{methoddesc}[Context]{newenvironment}{name\optional{, nargs\optional{, definition\optional{, opt}}}}
create a new \LaTeX\ environment like \macro{newenvironment}.  This works
exactly like the \method{newcommand()} method, except that the 
\var{definition} argument is a two element tuple where the first element
is a string containing the macro content to expand at the \macro{begin},
and the second element is the macro content to expand at the \macro{end}.

\textbf{Example}
\begin{verbatim}
c.newenvironment('mylist', 0, (r'\\begin{itemize}', r'\\end{itemize}'))
\end{verbatim}
\end{methoddesc}

\begin{methoddesc}[Context]{newif}{name\optional{, initial}}
create a new if like \macro{newif}.  This also creates macros corresponding
to \macro{\var{name}true} and \macro{\var{name}false}.
\end{methoddesc}

\begin{methoddesc}[Context]{newmuskip}{name\optional{, initial}}
create a new muskip like \macro{newmuskip}.
\end{methoddesc}

\begin{methoddesc}[Context]{newskip}{name\optional{, initial}}
create a new skip like \macro{newskip}.
\end{methoddesc}

\begin{memberdesc}[Context]{packages}
a dictionary of \LaTeX\ packages.  The keys are the names of the packages.
The values are dictionaries containing the options that were specified
when the package was loaded.
\end{memberdesc}

\begin{methoddesc}[Context]{pop}{\optional{obj}}
pop the top scope off of the stack.  If \var{obj} is specified, continue
to pop scopes off of the context stack until the scope that was originally
added by \var{obj} is found.  
\end{methoddesc}

\begin{methoddesc}[Context]{push}{\optional{context}}
add a new scope to the stack.  If a macro instance \var{context} is specified, 
the new scope's namespace is given by that object.
\end{methoddesc}

\begin{methoddesc}[Context]{ref}{obj, label}
set up a reference for resolution.  

\var{obj} is the macro object that is doing the referencing.

\var{label} is the label of the node that \var{obj} is looking for.

If the item that \var{obj} is looking for has already been labeled, 
the \member{idref} attribute of \var{obj} is set to the abject.  
Otherwise, the reference is stored away to be resolved later. 
\end{methoddesc}

\begin{methoddesc}[Context]{setVerbatimCatcodes}{}
set the current set of category codes to the set used for the verbatim
environment.
\end{methoddesc}

\begin{methoddesc}[Context]{whichCode}{char}
return the character code that \var{char} belongs to.  The category
codes are the same codes used by \TeX\ and are defined in the 
\class{Token} class.  
\end{methoddesc}

\subsection{Context language}\label{sec:context-language}

Contexts objects hold language information for the currently running
document. The current language is stored in
\var{Context.currentLanguage}. It can be changed from the \TeX\ source
using the babel package which invokes the \var{Context.loadLanguage}
method. New terms can be added in a user defined language file using the
lang-terms options (see Section~\ref{sec:config-document}). Languages
files are xml files. The following example should be self-explanatory.

\begin{verbatim}
<languages>
  <terms lang="fr" babel="french">
    <term name="proof">Démonstration</term>
  </terms>
</languages>
\end{verbatim}

This allows to add new terms which are then available to renderers in
the dictionary \var{Context.terms}. It also allows to override default
translations. For instance the above language file overwrites the
default translation of ``proof'' as ``Preuve'' in French.

\input{renderer-api}
\input{imager-api}

\appendix

\chapter{About This Document}

\input{about}

\chapter{Frequently Asked Questions}

\input{parsing-faq}

\printindex

\end{document}
